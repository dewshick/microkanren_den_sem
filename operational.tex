\documentclass{article}
\usepackage{amsmath, listings, amsthm, amssymb, proof, xspace}
\begin{document}

\section{Simple big step operational semantics}

We introduce nondeterministic big-step operational semantics for minikanren(without recursion and predicates, which will be added in next section, and disequality, which won't be here for a while).

$ x $ is a variable

$ t $ is a term, which can be variable or $ c_k(t_1, ... t_k) $  where $ c_k $ is k-ary constructor

$ g $ - goal, can be $ t_1 \equiv t_2 $ or $ g \vee g $ or $ g \wedge g $ or $ p(t_1 ... t_n) $

$ p $ -predicate symbol with $ \{x^p_i\}_{i\in (1:n)} $ variables and $ g_p $ body

$ s $ - substitution, from $ x $ to $ t $. Substitutions can be computed as most general unifiers of terms. Substitutions can be composed, resulting in most general substitution, which unifies terms unified by operands of composition.

Initial state $ s $ is empty substitution(identity) and initial goal statement $g$. Through the nondeterministic reduction steps we accumulate final substitution in state till we reach final substitution-answer.

\ 

\infer[\textsc{UNIFY}]{<t_1 \equiv t_2, s> \longrightarrow s \circ mgu(t_1, t_2)}{}

\ 


\infer[\textsc{DISJ-LEFT}]{<g_1 \vee g_2, s> \longrightarrow s'}{ <g_1, s> \longrightarrow s' }

\ 

\infer[\textsc{DISJ-RIGHT}]{<g_1 \vee g_2, s> \longrightarrow s'}{ <g_2, s> \longrightarrow s' }

\ 

\infer[\textsc{CONJ}]{ <g_1 \wedge g_2, s> \longrightarrow s_1 \circ s_2 }{ <g_1, s> \longrightarrow s_1 \qquad <g_2, s> \longrightarrow s_2 }

\newpage

\section{Fresh, predicates and semantic variables }

We introduce semantic environment $ \Gamma $ as mapping between syntacic($x$) and semantic($\sigma$) variables to avoid scoping issues. Semantic variables are expected to be elements of arbitrary countable set, they are used only to capture identity of variables. When $ \Gamma $ is applied to terms, we assume all the syntactic variables in terms are substituted to semantic variables.

Substitutions $ s $ now work on semantic domain instead of syntactic one. When substitution $ s $ is used in wider domain it is defined on, we assume it works as identity on variables, on which it is not defined.

\ 

\infer[\textsc{UNIFY-SEM}]{\Gamma \vdash <t_1 \equiv t_2,s> \longrightarrow s \circ mgu(\Gamma(t_1), \Gamma(t_2)) }{}

\ 

\infer[\textsc{FRESH}]{\Gamma \vdash <fresh(x)\ g, s> \longrightarrow s'|_{dom(s')/\sigma}  }{ \Gamma[x \rightarrow \sigma] \vdash <g, s> \longrightarrow s' }

\ 

\ 

Let $ \Gamma_{ext} = \Gamma[x^p_i \rightarrow \sigma_i]_{i \in (1:n)}, \sigma_i \notin dom(\Gamma) $

\ 

\infer[\textsc{PREDICATE}]{\Gamma \vdash <p(t_1,t_2 ... t_n), s> \longrightarrow (s' \circ (\bigcirc_{i \in (1:n)}\  mgu(\Gamma_{ext}(t_i), \Gamma_{ext}(x^p_i))))|_{dom(\Gamma)} }{ \Gamma_{ext} \vdash <g_p, s> \longrightarrow s' }


\section{Analogue of simple big step semantics, but for small-step}

We need to introduce success goal, to get rid of big-step-like reductions.

$ g $ - goal, can be $ t_1 \equiv t_2 $ or $ g \vee g $ or $ g \wedge g $ or $ p(t_1 ... t_n) $ or $ success $

\ 

\infer[\textsc{SUCCESS}]{<success, s> \longrightarrow s }{}

\ 

\infer[\textsc{UNIFY}]{<t_1 \equiv t_2, s> \longrightarrow < success, s \circ mgu(t_1, t_2) >}{}

\ 


\infer[\textsc{DISJ-LEFT}]{<g_1 \vee g_2, s> \longrightarrow <g_1, s>}{}

\ 

\infer[\textsc{DISJ-RIGHT}]{<g_1 \vee g_2, s> \longrightarrow <g_2, s>}{}

\ 

\infer[\textsc{CONJ}]{ <g_1 \wedge g_2, s> \longrightarrow < g_1'\wedge g_2', s_1 \circ s_2> }{ <g_1, s> \longrightarrow <g_1',s_1> \qquad <g_2, s> \longrightarrow <g_2',s_2> }

\ 

\infer[\textsc{CONJ-SUCCESS-LEFT}]{ <success \wedge g, s> \longrightarrow < g, s> }{}

\ 

\infer[\textsc{CONJ-SUCCESS-RIGHT}]{ < g \wedge success, s> \longrightarrow < g, s> }{}
\end{document}