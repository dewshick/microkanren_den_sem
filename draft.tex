\documentclass[fleqn]{article}
\usepackage{amsmath}
\usepackage{amssymb}
\usepackage{stmaryrd}
\usepackage{systeme}


\begin{document}
\newcommand{\keyword}[1]{\textrm{\textbf{#1}}}
\newcommand{\semantics}[1]{\llbracket#1\rrbracket}
\newcommand{\bigsemantics}[1]{\textbf{[[}#1\textbf{]]}}

\section{Syntax}

\subsection{Identifiers, constants and terms}

\[ A = \{a^k, ...\} \textrm{ - a finite set of atoms/constructors}; k \geq 0 \textrm{ stands for arity }\]
\[ \mathcal{D} = a^k(d_1,..d_k), d_i \in \mathcal{D} \textrm{ - ground terms(values).}\]
\[ X = \{ x,... \} \textrm{ - a set of syntacic variables } \]
\[ T_X = x \mid a_k(\mathbf{t}_1, ..., \mathbf{t}_k), \mathbf{t}_i \in T_X \textrm{ - set of terms with syntacic variables } \]
\[ n \in N \textrm{ - predicate names } \]

\subsection{Program syntax}

Program is a set of predicate definitions and a goal.
\[ P = D*; G \]

Predicate definition consist of predicate name, set of variables and body.
\[ D = N(X*)\ B \]

Goal definition consist of set of variables and body.
\[ G = (X*)\ B \]

Body can be unification between terms, conjunction, disjunction, predicate application and introduction of a temporary variable in body.
\[ b \in B, B = \keyword{fresh}(X)\ B  \mid B \vee B \mid B \wedge B \mid T_X \equiv T_X \mid N(X*) \]

\section{Unification}

Unification is a process of making two terms equal via syntactic substitution of variables\cite{baader90}. Syntactically-equal variables can be semantically-different, being introduced in different scopes. To avoid scoping issues, we introduce semantic variables, which will be used for unification, and correspond to syntactic variables in scope. Firstly we will define unification over terms with semantic variables, then show how to define one for syntactic terms.

\subsection{Semantic unification}

\[ \Sigma = \{ \sigma, ... \}\textrm{ - set of semantic variables }\]
\[ T_\Sigma = \sigma \mid a^k(t_1...t_k), t_i \in T \textrm{ - terms over \( \Sigma \) and A} \]
\[ \semantics{.}_{sem} \subset \mathcal{D}^\Sigma \]

We define semantics of expressions as the set of functions from variables to ground terms.
To explain the unification, first we define inductive extension of functions in \(\mathcal{D}^X\) to \(\mathcal{D}^{T_X}\)
\[ f \in \mathcal{D}^\Sigma, \overline{f} \in \mathcal{D}^{T_\Sigma} - \textrm{ - inductive extension of \( f \) on terms } \]
\[ \overline{f}(\sigma) = f(\sigma) \]
\[ \overline{f}(a^k(t_1...t_k)) = a^k(\overline{f}(t_1)...\overline{f}(t_k)) \]
\[ \semantics{ t_1 \equiv t_2 }_{sem} = \{ f \mid \overline{f}(t_1) = \overline{f}(t_2)  \} \]

\subsection{Syntactic unification}
\[ \Sigma_{def} \subset \Sigma \textrm{ - a set of defined semantic variables } \]
\[ \Gamma \in X \rightarrow \Sigma_{def} \textrm{ - mapping between syntactic and semantic variables }\]
\[ \overline{\Gamma} \in T_X \rightarrow T_{\Sigma} \textrm{ - inductive extension of \( \Gamma \) over terms } \]
\[ \semantics{.}_{syn}(\Gamma) \in 2^{D^\Sigma} \]
%%\[ \semantics{.}_{term}(\Gamma) \in T_X \rightarrow T_\Sigma \]

We won't have semantic variables and terms in syntax. To preserve scopes correctly, we need to separate syntactic and semantic variables, and introduce environment with correspondence of syntactic and semantic variables.
%%\[ \semantics{\mathbf{t}}_{term}(\Gamma) = \overline{\Gamma}(\mathbf{t}) \]
\[ \semantics{\mathbf{t}_1 \equiv \mathbf{t}_2}_{syn}(\Gamma) = \semantics{ \overline{\Gamma}(\mathbf{t}_1) \equiv \overline{\Gamma}(\mathbf{t}_2) }_{sem} \]

\newpage
%% what is e?
\section{Fresh}
\[ \semantics{\keyword{fresh}(x, b)}_{syn}(\Gamma) = \{ f|_{codom(\Gamma)} \mid f \in \semantics{e}_{syn}(\Gamma, x \rightarrow \sigma), \sigma \notin codom(\Gamma) \}\]

There's certain non-determinism introduced via adding unknown \(x\) to set of semantic variables in use, but it's uniqueness property \(x \notin X\) is enough to guarantee uniqueness of corresponding variable, introduced in this scope.

We evaluate fresh, introducing new semantic variable for correspodning syntactic variable, then we compute inner expression in extended environment and after that we shrink semantic domain of resulting functions to \(X\). With this approach, we can guarantee for top-level expression, that only free variables will be included in answer, also we guarantee the equality of domains for semantics of subexpressions, nested at the same level, say for \(\semantics{e_1 \wedge e_2}\). 

Also with this property we can safely reason about (extensional) equivalence of expressions.


\section{Conjunction/disjunction}
\[\semantics{e_1 \wedge e_2}_{syn}(\Gamma) = \semantics{e_1}_{syn}(\Gamma) \cap \semantics{e_2}_{syn}(\Gamma)\]
\[\semantics{e_1 \vee e_2}_{syn}(\Gamma) = \semantics{e_1}_{syn}(\Gamma) \cup \semantics{e_2}_{syn}(\Gamma)\]

We can safely use \(\cap\) and \(\cup\) since \(\semantics{e}(\Gamma) \in \mathcal{D}^\Sigma\).

\section{Predicates over terms} 
\subsection{Rules}

Abstraction rule:
\[\semantics{\{n_i(x_1, x_2, .. x_{k_i}), ...\ b_i\} }_{def} = \keyword{fix}\ g  \]
\[ \textrm{where } g_i(t_1,...t_k)(\Gamma) = (\semantics{b_i}_{syn}(\Gamma' ) \cap (\bigcap_{j \in (1:k)} \{ f | \overline{f}(\sigma_j) = \overline{f}(\overline{\Gamma(t_j)}) \} ))|_{codom(\Gamma)} \]
\[\textrm{where } \Gamma' = \Gamma, x_1 \rightarrow \sigma_1 ..., x_k \rightarrow \sigma_k; \sigma_i \notin codom(\Gamma) \]

Fixed point is well-defined, since we operate over functions over sets, that is, due to Tarski-Knaster theorem.

Application rule.
\[ \semantics{n_i(t_1, t_2 ... t_k)}_{syn}(\Gamma) = g_i(t_1, ... , t_k)(\Gamma) \]

\subsection{Relation semantics sanity proof}
\[ \semantics{ \keyword{fresh}(x_1, x_2 ... x_k)\ (e \wedge \bigwedge_{i \in [1 .. k]} (x_i \equiv t_i)) }_{syn}(\Gamma, \Omega) = \semantics{\keyword{let}\ p(x_1, ... ,x_k)\ e\ \keyword{in}\ p(t_1, ..., t_k)  } \]
(given that \( x_i \) do not occur in \( e \) and \( t_i \))

Proof:
\[ \semantics{ \keyword{fresh}(x_1, x_2 ... x_k)\ (e \wedge \bigwedge_{i \in [1 .. k]} (x_i \equiv t_i)) }_{syn}(\Gamma, \Omega) = \]
\[ \semantics{ e \wedge \bigwedge_{i \in [1 .. k]} (x_i \equiv t_i) }_{syn}(\Gamma')|_{codom(\Gamma)} \]
\[\textrm{where } \Gamma' = \Gamma, x_1 \rightarrow \sigma_1 ..., x_k \rightarrow \sigma_n, \sigma_i \notin codom(\Gamma) \]

\[ = (\semantics{e}_{syn}(\Gamma') \cap \bigcap \{ f \mid \overline{f}(\sigma_i) = \overline{f}(\overline{\Gamma}(t_i)) \})|_{codom(\Gamma)} \]
\[ \semantics{e}_{syn}(\Gamma') = \semantics{e}_{syn}(\Gamma) \textrm{ since \(x_i\) is not in \(e\) } \]
We can abstract out expression as relation application for relation \(p\), stored in \( \Omega \) and we'll have semantics of predicate exactly.
\subsection{Recursion}
We can define recursive relations via function from relation to relation. Function will be monothone (in set-lattice sense), since we only use union, intersection, projection and this three operations are monothone, their composition is monothone. 
Via Tarski-Knaster theorem, we can conclude existence of fixed point of function through its monotonicity. 
In the end it looks like this:
\[ goal \subset D^{codom(\Gamma)} \]
\[\semantics{\keyword{letrec}\ p(x_1, x_2, .. x_k)\ e_1\ \keyword{in}\ e_2}_{syn}(\Gamma) = \keyword{fix}(g)(\varnothing) \]
\[ \textrm{where } g(goal) = \semantics{\keyword{let}\ p(x_1, x_2, .. x_k)\ e_1\ \keyword{in}\ e_2}_{syn}(\Gamma, p \rightarrow goal) \]
%Dunno, how to connect it to environment and recursive calls properly, though.

%\subsection{Eigen}
%To define eigen-like-in-minikanren, we should modify semantics in 


\section{Unification Properties}
\subsection{Most General Unifier}

\[ \semantics{t_1 \equiv t_2}_{sem} \neq \varnothing \Longleftrightarrow \exists \  mgu(t_1, t_2)  \]

Direct implication:

Funcions in \( \semantics{t_1 \equiv t_2}_{sem} \) are unifiers by definition, therefore exist at least one unifier for terms, therefore \( mgu \) exists.

Backward implication:
\[\overline{mgu}(t_1, t_2)(t_1) = \overline{mgu}(t_1, t_2)(t_2) \Rightarrow \forall f \in D^\Sigma: \overline{f} \circ \overline{mgu}(t_1, t_2) \in \semantics{t_1 \equiv t_2}_{sem} \]


Stronger proposition:
\[ \semantics{ t_1 \equiv t_2 }_{sem} = \{ f \circ mgu(t_1, t_2) \mid f \in D^{\Sigma} \} \]

\( \subset \):
\[ \forall f \in \semantics{ t_1 \equiv t_2 }_{sem}\ :\ f = f' \circ mgu(t_1, t_2),\ f' \in T^\Sigma\]


\( f' \) will operate on terms with variables and produce terms with variables, but since end result will be a term without variables, \(f'\) eliminated all the variables, therefore was constant on the free variables of term under mgu. We can take any other function in \(D^\Sigma\), which is equal to \(f'\) on these free variables.



\( \supset \):


Since \( \forall f \in D^{\Sigma} f \circ mgu(t_1, t_2) \) is the same on \(t_1\) and \(t_2\), by definition of unification semantics, \( f \circ mgu(t_1, t_2) \in \semantics{ t_1 \equiv t_2 }_{sem}\).

\subsection{Subterms}
\subsubsection{ Unification of subterms is superset of unification of terms }


Proof by induction.

Induction base.
\[ f \in \semantics{ a_k(t_1, .. t_k) \equiv a_k(t_1',.., t_k') }_{sem} \]
\[ \Rightarrow_{def} \overline{f}(a_k(t_1, .. t_k)) = \overline{f}(a_k(t_1', .. t_k'))\]
\[ \Rightarrow a_k(\overline{f}(t_1), ... \overline{f}(t_k)) = a_k(\overline{f}(t_1'), ... \overline{f}(t_k')) \]
\[ \Rightarrow \forall i \in (1:k),\  \overline{f}(t_i) = \overline{f}(t_i') \]
\[ \Rightarrow f \in \semantics{ t_i \equiv t_i' }_{sem} \]

Induction step.

Statement is correct for terms, nested at level n, we should prove correctness for terms, nested at level n+1. We can apply induction base to parent of term at level n.

\subsubsection{ Unification functions over terms unify subterms }

Correct by previous statement.

\subsubsection{Correspondence between MGU and }

\subsection{ Properties  }

\end{document}
